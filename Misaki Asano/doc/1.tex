%第1章 まえがき

%1章1節
\section{研究背景および目的・目標}
近年では,IoT(Internet of Things)の普及が進んでおり,様々なものがインターネットに繋がる時代である.インターネトに繋がるものをIoT機器(IoTデバイス)といい,例として家電や自動車・産業用ロボットなどが挙げられる.
IoTを実現する上で,周囲の情報を検知するセンシングデバイスやネットワークは必要不可欠である.
そして,センシングデバイスがインターネットに繋がることで,外部の悪意のある第三者からの攻撃により,データを盗聴・改ざんする恐れが懸念される.

第三者からの攻撃事例に,タイヤ空気圧監視システムへの攻撃事例がある\cite{maegaki}.
タイヤ空気圧監視システム(TPMS:(Tire Pressure Monitoring System))とは,
タイヤの空気圧を常時監視するシステムであり,空気圧が低いタイヤで高速走行をすることによるタイヤバースト事故を防ぐ効果が期待される.
しかし,このTPMSの無線通信には脆弱性があることが問題となっている.
その問題の一つとして,TPMSでは通信メッセージが暗号化されていないため,盗聴解析が容易になるという点がある.
また, TPMSの空気圧報告メッセージになりすますことができるという問題点もある.
この事例から,センシングデバイスにおいて,ネットワーク上の脆弱性があることで外部からの攻撃によるデータの盗聴・改ざんが行われてしまうリスクがあることから,セキュリティ対策が不可欠となることが分かる.

センシングデバイスのセキュリティ対策の一つに認証・暗号化がある.
しかしながら,AESなど従来の暗号化方式は処理負荷が高いことから処理性能の低いセンシングデバイスには導入困難であるため,現在市場に出回っているセンシングデバイスには認証・暗号化といったセキュリティ対策が不十分で,脆弱性があると考えられる.
そこで,センシングデバイスなど,処理性能の低いIoT機器において,
極めて小さい処理負荷で認証と暗号化通信のできる軽量なセキュリティ対策が求められる.

本研究では,処理能力の低いセンシングデバイスで構成されるIoTシステムにおいて,デバイスとエッジサーバー間の
安全なデータ通信を行うセキュアな組込みシステムを開発することを目的とする.
具体的には,高知工学科大学の清水明宏教授が処理能力の低い装置へのセキュリティ機能実装のために
開発されたワンタイムパスワード認証方式SAS-L2をIoTセンシングデバイス
に実装することで,デバイス間の相互認証およびセンシングデータの暗号化通信方法を実現する.
なお,本研究は,2人チーム(浅野美咲,内山田隆太)でV字開発モデルに従って処理能力の低いセンシングデバイスで構成されるSAS-L2認証を導入したセキュアなIoTシステムの開発を行う.
システムの設計にはUMLを利用する.
チームメンバーの2人で分担し,内山田がエッジサーバーの実装,浅野がエッジデバイスの実装を行う.




%1章2節

\section{本論文の構成}
本論文の構成は以下の通りである.
第1章では,研究の背景および目的・目標について述べる.
第2章では,本論文に必要な暗号化手法と通信プロトコル,システム開発プロセスに関する用語について述べる.
第3章では,SAS-L2認証について述べる.
第4章では,センシングデータ通信の暗号化について述べる.
第5章では,開発したシステムの概要について述べる.
第6章では,システムの設計・テスト項目について述べる.
第7章では,実装したシステムの検証結果と考察を述べる.
第8章では,本論文のまとめを述べる.
