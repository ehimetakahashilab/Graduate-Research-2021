%あとがき

本研究の研究背景としてIoTデバイスに対するセキュリティ対策として認証・暗号化通信を行うことが挙げられるが,処理性能の低いIoTデバイスでは従来法の暗号方式での実装が困難である.
そこで,本研究ではワンタイムパスワード認証方式SAS-L2をIoTセンシングデバイス
に実装することで,デバイス間の認証およびセンシングデータの暗号化通信方法を実現することを目標に開発を行った.
SAS-L2認証方式を導入したIoTシステムの設計はUMLのユースケース図,クラス図,シーケンス図などを用いて行い,
その設計に基づきチームで分担し開発を行った.実装完了後は,V字開発モデルに従って,単体テスト,結合テスト,総合テストの順でテストを行った.
検証結果から,SAS-L2認証方式をセンシングデバイスに実装し,
デバイス間の認証とSAS-L2に基づいたデータ通信の暗号化ができたことが分かる.
また,センシングデータの暗号化では,排他的論理演算2回で済むため,処理能力の低いセンシングデバイスにおいてリアルタイムでの暗号化通信を実現することができる.
そして,問題点と課題として,センシングデバイスに初期認証情報や初期秘匿情報が書き込まれていない場合,
初期情報を取得する際の安全な方法の確立について,そして,認証情報と秘匿情報を不揮発性メモリに書き込む場合,
不揮発性メモリの書き込み回数を考慮して認証回数の決定をしなくてはならないという点が挙げられる.
それらを改善することで,本システムでの処理性能の低いセンシングデバイスに認証方式と暗号化通信を導入することによる,さらなるセキュリティ強化に繋がると考える.
